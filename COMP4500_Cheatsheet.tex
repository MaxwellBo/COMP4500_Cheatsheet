\documentclass[fontsize=10pt,a4paper]{article}
\usepackage{amsmath} % Essential math and alignment (use `&` to align operators')
% Google for more info
\usepackage{amssymb} % Essential symbols for sets and stuff

\usepackage[margin=0.3in]{geometry} % Essential document config

\usepackage[compact]{titlesec}
\titleformat*{\section}{\normalsize\bfseries\scshape}
\titleformat*{\subsection}{\small\bfseries}
%\usepackage{sectsty}
%\sectionfont{\fontsize{7}{9}\selectfont}
%\subsectionfont{\fontsize{6}{8}\selectfont}
%\subsubsectionfont{\fontsize{5}{7}\selectfont}

\usepackage{multicol} % Imports the column
\setlength{\columnseprule}{0.4pt} % Divide size

\usepackage{dirtree}

\begin{document}
\begin{multicols}{3}

    \small

    \section{Hierarchy}
    $$O(1) \leqslant O(log(n)) \leqslant O(n^c)$$
    $$\leqslant O(n) \leqslant O(n^2)$$
    $$\leqslant O(n^c) \leqslant O(c^n)$$


    \section{Big-O}
    We say that $f(n)$ grows asymptotically no faster than $g(n)$ if there is a constant $c_1>0$ such that $f(n)\leq c_1\cdot g(n)$ and holds for all $n$ at least a constant $c_2$. This is denoted by $f(n)=O(g(n))$.\\
    $\lim_{n\rightarrow\infty}\frac{f(n)}{g(n)}=c$ for some constant $c$
    \subsection{Example}
    $1000\log_2n=O(n)$,\\$n\neq O(10000\log_2n)$\\
    $\log_{b_1}n=O(\log_{b_2}n)$ for any constants $b_1>1$ and $b_2>1$.
    Therefore $f(n)=2+6\log_2n$ can be represented; $f(n)=O(\log n)$
    \section{Big-$\Omega$}
    If $g(n)=O(f(n))$, then $f(n)=\Omega(g(n))$ to indicate that $f(n)$ grows asymptotically no slower than $g(n)$. We say that $f(n)$ grows asymptotically no slower than $g(n)$ if $c_1>0$ such $f(n)\geq c_1\cdot g(n)$ for $n>c_2$; denoted by $f(n)=\Omega(g(n))$
    \section{Big-$\Theta$}
    If $f(n)=O(g(n))$ and $f(n)=\Omega(g(n))$, then $f(n)=\Theta(g(n))$ to indicate that $f(n)$ grows asymptotically as fast as $g(n)$

    When using `Direction 1: Constant Finding' setting $c_1$, always set it to match the coefficent on the LHS so that you can cancel.\\
    When trying to get a contradiction, try and isolate an $x \cdot c_1$ on the RHS, where $x \in \mathbb{Z}$, such that an expression that contains $n$ is $\leqslant x \cdot c_1$\\
    Make judicious use of the $max$ function when adding functions together
    If $f_1(n) + f_2(n) \leqslant c_1 \cdot g_1(n) +c'_1 \cdot g_2(n) \leqslant max\{c_1 , c'_1 \} \cdot (g_1(n) + g_2(n))$, for all $n \geqslant max\{c_2, c'_2\}$.\\

    \section{The Master Theorem}
    \subsection{Theorem 1}
    $n+\frac{n}{c}+\frac{n}{c^2}+\ldots+\frac{n}{c^h}=O(n)$
    \subsection{Theorem 2}
    Let $f(n)$ be a function that returns a positive value for every integer $n>0$. We know:
    \begin{align*}
        f(1) & \leqslant c_1\\
        f(n) & \leqslant \alpha \cdot f(\lceil n / \beta \rceil) + c_2 \cdot n^{\gamma} \text{ for } n \geqslant 2
    \end{align*}
    where $\alpha, \beta, \gamma, c_1$ and $c_2$ are positive constants. Then:
    \begin{itemize}
        \item If $log_{b} \alpha < \gamma$ then $f(n) = O(n^\gamma)$
        \item If $log_{b} \alpha = \gamma$ then $f(n) = O(n^\gamma \cdot log(n))$
        \item If $log_{b} \alpha > \gamma$ then $f(n) = O(n^{log_\beta(a)})$

    \end{itemize}


\end{multicols}

\end{document}

